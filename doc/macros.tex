\makeatletter

\newcommand\pagelabel[1]{\phantomsection\label{#1}}
\newcommand\phantomlabel[1]{\phantomsection\label{#1}}

\DeclareRobustCommand\texinput[1]{\textbf{How to input:} \texttt{\def\\{\textbackslash}#1}}

\DeclareRobustCommand\sagelink[1]{%
  \let\SAGECELLLINK\undefined
  \InputIfFileExists{#1.sagelink}{}{}%
  \typeout{XXX:\meaning\SAGECELLLINK}%
  \ifx\SAGECELLLINK\undefined\expandafter\@firstoftwo
  \else\expandafter\@secondoftwo\fi
  {\TODOQ{Create #1.sagelink}}%
  {\edef\@tempa{\noexpand\href{\SAGECELLLINK}}%
    \@tempa}}

  \mdfdefinestyle{proofstyle}{%
    backgroundcolor=black!4!white,
    innertopmargin=3pt,
    innerbottommargin=3pt,
    innerleftmargin=3pt,
    innerrightmargin=3pt,
    linewidth=0pt
  }

  \mdfdefinestyle{claimproofstyle}{%
    backgroundcolor=black!8!white,
    innertopmargin=0pt,
    innerbottommargin=3pt,
    innerleftmargin=3pt,
    innerrightmargin=3pt,
    leftmargin=10pt,
    rightmargin=10pt,
  %  linewidth=0pt
  }

  \newenvironment{proof}[1][]{%
    \begin{mdframed}[style=proofstyle]%
      \begin{trivlist}%
      \item \textit{Proof\ifx\proof#1\proof\else\ #1\fi. }%
      }{
        \qed\global\let\qed\qedmacro
      \end{trivlist}
    \end{mdframed}}%

  \newenvironment{claimproof}{%
    \let\qedbox\qedboxclaim
    \begin{mdframed}[style=claimproofstyle]%
      \begin{trivlist}%
      \item \textit{Proof of claim. }%
      }{
        \qed\global\let\qed\qedmacro
      \end{trivlist}
    \end{mdframed}}%

  \newenvironment{claimproofstart}{%
    \let\qedbox\qedboxclaim
    \begin{mdframed}[style=claimproofstyle]%
      \begin{trivlist}%
      \item \textit{Proof of claim. }%
      }{
      \end{trivlist}
    \end{mdframed}}%

  \newenvironment{claimproofend}{%
    \let\qedbox\qedboxclaim
    \begin{mdframed}[style=claimproofstyle]%
      \begin{trivlist}%
      \item%
      }{
        \qed\global\let\qed\qedmacro
      \end{trivlist}
    \end{mdframed}}%




% \showafter{year}{month}{day}{text} evaluates to {text} starting from year/month/day
\newcommand\showafter[4]{%
  \ifnum\year>#1\relax
    #4% today.year > args.year
  \else
    \ifnum\year=#1\relax
    % today.year = args.year
      \ifnum\month>#2\relax
      % today.month > args.month
        #4%
      \else
        \ifnum\month=#2\relax
        % today.month = args.month
          \ifnum\day>#3\relax
          % today.day >= args.day
            #4%
          \else
            \ch@removeduplicatespace
          \fi
        \else
          \ch@removeduplicatespace
        \fi
      \fi
    \else
      \ch@removeduplicatespace
    \fi
  \fi}


\let\textttOLD\texttt
\def\texttt#1{\textttOLD{\frenchspacing #1}}

\makeatletter
\newcommand\qedboxclaim{\ensuremath\diamond}
\newcommand\qedbox{\m@th$\Box$}
\newcommand\qedmacro{\hfill\qedbox}
\global\let\qed\qedmacro
\newcommand\skipqed{\global\let\qed\empty}
\newcommand\mathQED{\tag*{\qedbox}\skipqed}

\newcommand\txtrel[1]{\stackrel{\hskip-1in\text{\tiny#1}\hskip-1in}}
\newcommand\eqrefrel[1]{\txtrel{\eqref{#1}}}
\newcommand\starrel{\txtrel{$(*)$}}
\newcommand\tristarrel{\txtrel{$\scriptscriptstyle (\!*\!*\!*\!)$}}
\newcommand\fourstarrel{\txtrel{$\scriptscriptstyle (\!*\!*\!*\!*\!)$}}
\newcommand\starstarrel{\txtrel{$(**)$}}
\newcommand\boardpic[1]{\par{\centering\includegraphics[keepaspectratio=true,width=\hsize,height=2.5in]{#1}\par}}
\newcommand\rimplies{\DOTSB\;\Longleftarrow\;}
\newcommand\sublemmaref[2]{\autoref{#1}\,\eqref{sub#1:#2}}
\DeclareRobustCommand\autorefshort[1]{%
  \def\lemmaautorefname{Lem.\!}%
  \autoref{#1},
  p.\,\pageref{#1}}
\newcommand\bit{\{0,1\}}
\newcommand\bits[1]{\bit^{#1}}

\def\parenthesis@resizecommand@open{}
\def\parenthesis@resizecommand@close{}
\def\parenthesis@#1#2#3#4#5{%
  \global\let\parenthesis@resizecommand@open\empty
  \global\let\parenthesis@resizecommand@close\empty
  #1#3%
  {#5}%
  #2#4%
}
\DeclareRobustCommand\parenthesis[3]{%
  \expandafter\expandafter\expandafter
  \parenthesis@\expandafter\expandafter\expandafter{\expandafter\parenthesis@resizecommand@open\expandafter}\expandafter{\parenthesis@resizecommand@close}{#1}{#2}{#3}}


% #1=bigl #2=bigr #3=( #4=txt #5=) #6=txt #7=( #8=txt #9=)
\def\double@parenthesis@#1#2#3#4#5#6#7#8#9{%
  \global\let\parenthesis@resizecommand@open\empty
  \global\let\parenthesis@resizecommand@close\empty
  #1#3{#4}#2#5%
  {#6}%
  #1#7{#8}#2#9%
}

\DeclareRobustCommand\doubleParenthesis[7]{%
  \expandafter\expandafter\expandafter
  \double@parenthesis@\expandafter\expandafter\expandafter{\expandafter\parenthesis@resizecommand@open\expandafter}\expandafter{\parenthesis@resizecommand@close}{#1}{#2}{#3}{#4}{#5}{#6}{#7}}


\DeclareRobustCommand\resize@next@paren[2]{%
  \gdef\parenthesis@resizecommand@open{#1}%
  \gdef\parenthesis@resizecommand@close{#2}%
}

\newcommand\paren{\parenthesis()}
\newcommand\braces{\parenthesis\{\}}

\newcommand\pb{\resize@next@paren\bigl\bigr}
\newcommand\pbb{\resize@next@paren\biggl\biggr}
\newcommand\plr{\resize@next@paren\left\right}
\newcommand\pB{\resize@next@paren\Bigl\Bigr}

\newlistof{rule}{rule}{{List of rules}}
\renewcommand\cftruletitlefont{\section*}
\newcommand\rulestrut{}
\newcommand\rulespace{}
\newcommand\RULE[4][]{%
  \inferrule[\rulestrut{#2\ifx\norulelemmalink\undefined
    \textsuperscript{\normalfont\tiny[\autorefshort{rule-lemma:#2}]}\fi}%
    \refstepcounter{rule}%
    \label{rule:#2}%
    \index{#2@\textsc{#2} (rule)}%
    \addcontentsline{rule}{rule}{\textsc{#2}%
      \ifx\@nil#1\@nil\else\space -- #1\fi}
  ]{#3}{#4}\rulespace%  
}
\newcommand\RULEX[4][]{%
  \inferrule[\rulestrut{#2}%
      \ifx\@nil#1\@nil\else\space -- #1\fi]{#3}{#4}\rulespace%  
}
\newenvironment{ruleblock}{%
  \begin{center}%
    \def\rulestrut{\vrule height 1.3\baselineskip width 0pt\relax}%
    \def\rulespace{\qquad}%
    \noindent
  }{%
  \end{center}}
  
\newcommand\refrule[1]{\hyperref[rule:#1]{\hbox{\textsc{#1}}} rule}
\newcommand\ruleref[1]{\hyperref[rule:#1]{rule \hbox{\textsc{#1}}}}
\newcommand\rulerefx[1]{\hyperref[rule:#1]{\hbox{\textsc{#1}}}}
\newcommand\Ruleref[1]{\hyperref[rule:#1]{Rule \hbox{\textsc{#1}}}}


%\newcommand\butter[2]{\basis{#1}\bra{#2}}
%\newcommand\selfbutter[2]{\proj{\basis{#1}{#2}}}

%\declare{macro=\varcard, code=\kappa, 
%  description={Upper bound for cardinality of variable types}}

\declare{macro=\abs, argspec=[1], code=\parenthesis\lvert\rvert{#1},
  placeholder=\abs x, nopage=true,
  description=Absolute value of $x$ / cardinality of set $x$}
%\declare{macro=\babs, argspec=[1], code={\bigl\lvert#1\bigr\rvert}, 
%  variantof=\abs}

\declare{macro=\qu, argspec=[1], code=#1^\mathsf{qu},
  placeholder=\qu V,
  description={Quantum variables in $V$}}
\declare{macro=\cl, argspec=[1], code=#1^\mathsf{cl},
  placeholder=\cl V,
  description={Classical variables in $V$}}

\declare{macro=\fv, code=\mathit{fv},
  placeholder=\fv(e),
  description={Free variables in an expression $e$ (or program)}}

\declare{macro=\overwr, code=\mathit{overwr},
  placeholder=\overwr\bc,
  description={Overwritten variables in program $\bc$}}

\declare{macro=\typev, argspec=[1], code=\mathsf{Type}_{#1},
  placeholder=\typev v,
  description={Type of variable $v$}}
\declare{macro=\types, argspec=[1], code=\mathsf{Type}^\mathsf{set}_{#1},
  placeholder=\types X,
  description={Type of a set $V$ of variables}}
\declare{macro=\typel, argspec=[1], code=\mathsf{Type}^\mathsf{list}_{#1},
  placeholder=\typel V,
  description={Type of a list $V$ of variables}}
\declare{macro=\typee, argspec=[1], code=\mathsf{Type}^\mathsf{exp}_{#1},
  placeholder=\typee e,
  description={Type of an expression $e$}}

\declare{macro=\denotee, argspec=[2], code=\parenthesis\llbracket\rrbracket{#1}_{#2},
  placeholder=\denotee em,
  description={Denotation of a classical expression $e$, evaluated on classical memory $m$}}

%\declare{macro=\denoteex, argspec=[1], code=\llbracket#1\rrbracket,
%  variantof=\denotee}

\declare{macro=\bc, code=\mathbf c, nopage=true,
  variable=true,
  description={A program}}
\declare{macro=\bd, code=\mathbf d, nopage=true,
  variable=true,
  description={A program}}
\newcommand\be{\mathbf e}

\declare{macro=\denotc, argspec=[1], code={\parenthesis\llbracket\rrbracket{#1}},
  placeholder={\denotc \bc},
  description={Denotation of a program $\bc$}}
\declare{macro=\denotcl, argspec=[1], code={\parenthesis\llbracket\rrbracket{#1}_\mathbf{class}},
  placeholder={\denotcl \bc},
  description={Classical denotation of a program $\bc$}}

\declare{macro=\Skip, code=\mathbf{skip},
  description={Program that does nothing}}

\declare{macro=\langif, argspec=[3], code={\mathbf{if}\ #1\ \mathbf{then}\ #2\ \mathbf{else}\ #3},
  placeholder=\langif e{c_1}{c_2},
  description=Statement: If (conditional)}
%\declare{macro=\langifx, argspec=[2], code={\mathbf{if}\ #1\ \mathbf{then}\ #2},
%  variantof=\langif}
%\declare{macro=\langifkw, code={\mathbf{if}},
%  variantof=\langif}

\declare{macro=\while, argspec=[2], code={\mathbf{while}\ #1\ \mathbf{do}\ #2},
  placeholder=\while ec,
  description=Statement: While loop}
%\declare{macro=\whilekw, code={\mathbf{while}},
%  variantof=\while}
\newcommand\whilep[2]{\while{#1}{\{#2\}}}

\declare{macro=\sample, argspec=[2], code={#1\stackrel{{\scriptstyle{}\$}}\leftarrow #2},
  placeholder=\sample \xx e,
  description={Statement: Sample $\xx$ according to distribution $e$}}

\declare{macro=\Qinit, argspec=[2],
  code={#1\stackrel{{\scriptstyle{\text{\bfseries\sffamily q}}}}\leftarrow #2},
  % code={\mathbf{init}\ #1\leftarrow #2},
  % code={#1\twoheadleftarrow#2},
  placeholder=\Qinit{\qq_1\dots \qq_n}{e},
  description={Statement: Initialize $\qq_1,\dots,\qq_n$ with quantum state $e$}}
\declare{macro=\Qapply, argspec=[2], code={\mathbf{apply}\ #1\ \mathbf{to}\ #2},
  placeholder=\Qapply{\qq_1\dots \qq_n}{U},
  description={Statement: Apply unitary/isometry $U$ to quantum registers $\qq_1\dots \qq_n$}}

\declare{macro=\Qmeasure, argspec=[3], code={#1 \leftarrow \mathbf{measure}\ #2\ \mathbf{with}\ #3},
  placeholder=\Qmeasure{\xx}{\qq_1\dots\qq_n}{e},
  description={Statement: Measure quantum variables $\qq_1\dots\qq_n$ with measurement $e$}}

\declare{macro=\bool, code=\mathbb B,
  description={Booleans. $\bool=\{\true,\false\}$}}

\declare{macro=\elltwo, argspec=[1], code={\ell^2(#1)},
  placeholder=\elltwo B,
  description={Hilbert space with basis indexed by $B$}}
\declare{macro=\elltwov, argspec=[1], code={\ell^2\parenthesis[]{#1}},
  placeholder=\elltwov V,
  description={$\elltwo{\types V}$ -- Hilbert space with basis $\types V$}}

\declare{macro=\ellone, argspec=[1], code={\mathbf D(#1)},
  placeholder=\ellone B,
  description={Distributions on $B$}}
\declare{macro=\ellonev, argspec=[1], code={\mathbf D[#1]},
  placeholder=\ellonev V,
  description={$\ellone{\types V}$ -- Distributions on $\types V$}}

\declare{macro=\iso, argspec=[1], code=\mathbf{Iso}(#1),
  placeholder={\iso{X,Y}, \iso X},
  description={Isometries from $\elltwo X$ to $\elltwo Y$ (on $\elltwo X$)}}
\declare{macro=\isov, argspec=[1], code={\mathbf{Iso}[#1]},
  placeholder={\isov{V,W}, \iso V},
  description={Isometries from $\elltwov V$ to $\elltwov W$ (on $\elltwov V$)}}

\declare{macro=\uni, argspec=[1], code=\mathbf{U}(#1),
  placeholder={\uni{X,Y}, \uni X},
  description={Unitaries from $\elltwo X$ to $\elltwo Y$ (on $\elltwo X$)}}
\declare{macro=\univ, argspec=[1], code={\mathbf{U}[#1]},
  placeholder={\univ{V,W}, \uni V},
  description={Unitaries from $\elltwov V$ to $\elltwov W$ (on $\elltwov V$)}}

\declare{macro=\distr, argspec=[1], code=\mathbf D^{\leq1}(#1),
  placeholder=\distr X,
  description={Sub-probability distributions over $X$}}
\declare{macro=\distrv, argspec=[1], code=\mathbf D^{\leq1}[#1],
  placeholder=\distrv V,
  description={Sub-probability distributions over variables $V$}}

\declare{macro=\setC, code=\mathbb C,
  description={Complex numbers}}
\declare{macro=\setR, code=\mathbb R,
  description={Real numbers}}
\declare{macro=\setRpos, code=\mathbb R_{\geq0},
  description={Non-negative real numbers}}

\declare{macro=\tracecl, argspec=[1], code=\mathbf T(#1),
  placeholder={\tracecl X},
  description=Trace class operators on $\elltwo X$}
\declare{macro=\tracepos, argspec=[1], code=\mathbf T^+(#1),
  placeholder={\tracepos X},
  description=Positive trace class operators on $\elltwo X$}
\declare{macro=\tensor, code=\otimes,
  placeholder={A\otimes B},
  description={Tensor product of vectors/operators $A$ and $B$}}

\declare{macro=\traceclv, argspec=[1], code=\mathbf T[#1],
  placeholder={\traceclv V},
  description=Trace class operators on $\elltwov V$}
\declare{macro=\traceposv, argspec=[1], code=\mathbf T^+[#1],
  placeholder={\traceposv V},
  description=Positive trace class operators on $\elltwov V$}
\declare{macro=\traceclcq, argspec=[1], code=\mathbf T_{\mathit{cq}}[#1],
  placeholder={\traceclcq V},
  description=Trace class cq-operators on $\elltwov V$}

\declare{macro=\pointdistr, argspec=[1],
  code={\delta_{#1}},
  placeholder=\pointdistr{x},
  description={Point distribution: returns $x$ with probability $1$}}

\declare{macro=\restricte, argspec=[1], code=\mathord\downarrow_{#1},
  placeholder=\restricte e(\rho),
  description={Restrict state/distribution $\rho$ to the case $e=\true$ holds}}
\newcommand\restrictno[2]{#1}

\declare{macro=\seq, argspec=[2], code=#1;#2,
  placeholder=\bc;\bd,
  description={Sequential composition of programs}}


% \declare{macro=\pointstate, argspec=[2],
%   code={\delta_{#1}(#2)},
%   placeholder=\delta_V(x),
%   description={Point distribution as density operator}}
\newcommand\pointstate[2]{\proj{\basis{#1}{#2}}}

\declare{macro=\true, code=\mathtt{true}, nopage=true,
  description={Truth value ``true''}}
\declare{macro=\false, code=\mathtt{false}, nopage=true,
  description={Truth value ``false''}}

\declare{macro=\traceposcq, argspec=[1], code=\mathbf T^+_{\mathit{cq}}[#1],
  placeholder={\traceposcq V},
  description=Positive trace class cq-operators on $\elltwov V$}

\declare{macro=\qlift, argspec=[1], code=\mathsf{lift}\paren{#1},
  placeholder=\qlift\mu,
  description=Transforms a distribution $\mu$ into a density operator}

\declare{macro=\lift, argspec=[2], code={#1\textrm\guillemotright#2},
  placeholder=\lift AQ,
  description={Lifts operator or subspace to variables $Q$}}

%\declare{macro=\compl, argspec=[1], code=#1^{\complement},
%  placeholder=\compl X,
%  description=Complement of the set $X$}

\declare{macro=\calE, code=\mathcal E, variable=true,
  description=A superoperator}

\declare{macro=\bounded, argspec=[1], code=\mathbf B(#1),
  placeholder={\bounded{X,Y}},
  description=Bounded linear operators from $\elltwo X$ to $\elltwo Y$}
\declare{macro=\boundedleq, argspec=[1], code=\mathbf B^{\leq1}(#1),
  placeholder={\boundedleq{X,Y}},
  description=Bounded linear operators with operator norm $\leq1$}
\declare{macro=\boundedv, argspec=[1], code=\mathbf B[#1],
  placeholder=\boundedv{V,W},
  description={Bounded linear operators from $\elltwov V$ to $\elltwo W$}}

\declare{macro=\idv, argspec=[1], code=\mathit{id}_{#1},
  placeholder=\idv V,
  description=Identity on $\elltwov V$ or on $\boundedv V$}
\declare{macro=\id, code=\mathit{id},
  description=Identity}

\declare{macro=\adj, argspec=[1], code=#1^*,
  placeholder={\adj A},
  description={Adjoint of the operator $A$}}

%\newcommand\basisOLD[1]{\lvert #1\rangle}
\declare{macro=\basis, argspec=[2], code={\parenthesis\lvert\rangle{#2}_{\!\scriptscriptstyle{#1}}},
  placeholder={\basis{} b, \basis V b},
  description={Basis vector in Hilbert space $\elltwov V$}}
\newcommand\ket{\basis{}}
%\declare{macro=\bra, argspec=[1], code={\langle #1\lvert},
%  placeholder=\bra b,
%  description={$\adj{\basis b}$, adjoint of basis vector $\basis b$}}

% \declare{macro=\hoarev, argspec=[4], code={\{#1\}#2\{#3\}^{#4}},
%   placeholder=\hoarev F\bc GV,
%   description={Possibilistic Hoare judgement (variables $V$ omitted if clear)}}
% \declare{macro=\hoare, argspec=[3], code={\{#1\}#2\{#3\}},
%   variantof=\hoarev}

\declare{macro=\rhl, argspec=[4],
  % code={\{#1\}#2\pmb\sim#3\{#4\}},
  code={\doubleParenthesis\{{#1}\}{#2\pmb\sim#3}\{{#4}\}},
  placeholder=\rhl F\bc{\bc'} G,
  description={Quantum relational Hoare judgment}}
\declare{macro=\rhlfirst, argspec=[4],
  % code={\{#1\}#2\pmb\sim#3\{#4\}},
  code={\doubleParenthesis\{{#1}\}{#2\pmb\sim#3}\{{#4}\}_\mathsf{nonsep}},
  placeholder=\rhlfirst F\bc{\bc'} G,
  description={qRHL judgment, non-separable definition}}
\declare{macro=\rhlunif, argspec=[4],
  % code={\{#1\}#2\pmb\sim#3\{#4\}},
  code={\doubleParenthesis\{{#1}\}{#2\pmb\sim#3}\{{#4}\}_\mathsf{uniform}},
  placeholder=\rhlunif F\bc{\bc'} G,
  description={qRHL judgment, uniform definition}}
%\declare{macro=\rhlv, argspec=[6], code={\{#1\}#2\sim#3\{#4\}^{#5,#6}},
%  variantof=\rhlv}
\declare{macro=\prhl, argspec=[4],
  % code={\{#1\}#2\pmb\sim#3\{#4\}},
  code={\doubleParenthesis\{{#1}\}{#2\pmb\sim#3}\{{#4}\}_\mathsf{class}},
  placeholder=\prhl F\bc{\bc'} G,
  description={pRHL judgement (classical)}}

\declare{macro=\partr, argspec=[2], 
  code=\mathord{\operatorname{tr}^{\scriptscriptstyle[#1]}_{\scriptscriptstyle #2}},
  placeholder=\partr VW(\rho),
  description={Partial trace, keeping variables $V$, dropping variables $W$}}

\declare{macro=\im, code=\operatorname{im},
  placeholder=\im A,
  description={Image of $A$}}
\declare{macro=\dom, code=\operatorname{dom},
  placeholder=\dom f,
  description={Domain of $f$}}

\declare{macro=\norm, argspec=[1], code={\parenthesis\lVert\rVert{#1}},
  placeholder=\norm x, description={$\ell_2$-norm of vector $x$, or operator-norm}}
%\declare{macro=\bnorm, argspec=[1], code={\bigl\lVert#1\bigr\rVert}, variantof=\norm}

\declare{macro=\tr, code=\operatorname{tr},
  placeholder=\tr M,
  description={Trace of matrix/operator $M$}}

\declare{macro=\CL, argspec=[1], code=\mathfrak{Cla}\parenthesis[]{#1},
  placeholder=\CL e,
  description={Classical predicate meaning $e=\true$}}

\declare{macro=\SPAN, code={\operatorname{span}},
  placeholder=\SPAN A,
  description={Span, smallest subspace containing $A$}}
\newcommand\SPANO[1]{\SPAN\{#1\}}

\declare{macro=\quanteq, code={\equiv_\mathsf{quant}},
  placeholder=X_1\quanteq X_2,
  description={Equality of quantum variables $X_1$ and $X_2$}}

\declare{macro=\proj, argspec=[1], code={\mathsf{proj}\parenthesis(){#1}},
  placeholder=\proj x,
  description={Projector onto $x$, i.e., $x\adj x$}}

\declare{macro=\prafter, argspec=[3], code={\Pr\parenthesis[]{#1 : #2(#3)}},
  placeholder=\prafter e\bc\rho,
  description={Probability that $e$ holds after running $\bc$ on initial state $\rho$}}

\declare{macro=\Urename, argspec=[1], code={U_{\mathit{rename},#1}},
  placeholder=\Urename{\sigma},
  description={Unitary: Renames variables according to bijection $\sigma$}}


\declare{macro=\Erename, argspec=[1], code={\calE_{\mathit{rename},#1}},
  placeholder=\Erename{\sigma},
  description={cq-superoperator: Renames variables according to bijection $\sigma$}}

\declare{macro=\subst, argspec=[2], code=#1#2,
  placeholder={\subst e\sigma, \subst\bc\sigma},
  description={Apply variable renaming $\sigma$ to expression $e$}}

\declare{macro=\substi, argspec=[2], code=#1\{#2\},
  placeholder=\substi e{f/\xx},
  description={Substitute $f$ for variable $\xx$ in $e$}}

\declare{macro=\restrict, argspec=[2], code=#1|_{#2},
  placeholder=\restrict fM,
  description={Restriction of function $f$ to domain $M$}}

\declare{macro=\xx, code=\mathbf x,
  description={A classical program variable}}
\declare{macro=\yy, code=\mathbf y,
  description={A classical program variable}}
%\declare{macro=\zz, code=\mathbf z,
%  description={A classical program variable}}
\declare{macro=\qq, code=\mathbf q,
  description={A quantum program variable}}
\newcommand\rr{\mathbf r}
% \declare{macro=\rr, code=\mathbf r,
%   description={A quantum program variable}}


\declare{macro=\upd, argspec=[3], code={#1(#2:=#3)},
  placeholder=\upd fxy,
  description={Function update, i.e., $(\upd fxy)(x)=y$}}


\declare{macro=\assign, argspec=[2], code={#1\leftarrow #2},
  placeholder=\assign xe,
  description={Program: assigns expression $e$ to $x$}}

\declare{macro=\suppd, code=\operatorname{supp},
  placeholder=\suppd \mu,
  description={Support of distribution $\mu$}}
\declare{macro=\suppo, code=\operatorname{\mathbf{supp}},
  placeholder=\suppo M,
  description={Support of an operator $M$}}

\declare{macro=\marginal, argspec=[2], code={\mathsf{marginal}_{#1}\parenthesis(){#2}},
  placeholder=\marginal i\mu,
  description={$i$-th marginal distribution of $\mu$ (for $\mu\in\distr{X\times Y}$, $i=1,2$)}}

\declare{macro=\orth, argspec=[1], code={#1^\bot},
  placeholder=\orth S,
  description={Orthogonal complement of subspace $S$}}

% \declare{macro=\calH, code=\mathcal H,
%   description={A Hilbert space}}

% \declare{macro=\pinv, argspec=[1], code=#1^+,
%   placeholder=\pinv A,
%   description={Moore-Penrose pseudoinverse}}

\declare{macro=\idx, argspec=[1], code=\operatorname{idx}_{#1},
  placeholder={\idx 1\bc, \idx 1 e},
  description={Add index $1$ to every variables in $\bc$ or $e$}}

\declare{macro=\Meas, argspec=[2], code={\mathbf{Meas}(#1,#2)},
  placeholder=\Meas DE,
  description={Projective measurements on $\elltwo E$ with outcomes in $D$}}

\declare{macro=\Uvarnames, argspec=[1], code=U_{\mathit{vars},#1},
  placeholder=\Uvarnames{Q},
  description={Canonical isomorphism between $\elltwo{\typel Q}$
    and $\elltwov{\types Q}$ for a list $Q$}}

% \declare{macro=\leftim, argspec=[1], code=\overleftarrow\im\, #1,
%   placeholder=\leftim R,
%   description={Left image of relation $R$}}

% \declare{macro=\rightim, argspec=[1], code=\overrightarrow\im\, #1,
%   placeholder=\rightim R,
%   description={Right image of relation $R$}}

\declare{macro=\spaceat, argspec=[2], code=#1\div#2,
  placeholder=\spaceat A\psi,
  description={Part of $A$ containing $\psi$}}

\declare{macro=\memuni, argspec=[1], code=#1,
  placeholder=\memuni{m_1m_2},
  description={Union (concatenation) of memories $m_1$, $m_2$}}

\declare{macro=\powerset, argspec=[1], code=2^{#1},
  placeholder=\powerset M,
  description={Powerset of $M$}}

%\declare{macro=\TOOLotimes,
%  placeholder={\otimes},
%  description={Tensor product (Isabelle/HOL constant)}}

\declare{macro=\TOOLtensoro, code={\otimes_o},
  placeholder={A \TOOLtensoro B},
  description={Tensor product of two operators/matrices}}

\declare{macro=\TOOLtensorl, code={\otimes_l},
  placeholder={A \TOOLtensorl B},
  description={Tensor product of two vectors/quantum states}}

\declare{macro=\TOOLtensorS, code={\otimes_S},
  placeholder={A \TOOLtensorS B},
  description={Tensor product of two subspaces}}

\declare{macro=\TOOLoCL, code={\circ_{CL}}
  placeholder={A \TOOLoCL B},
  description={Composition (product) of operators/matrices}}

\declare{macro=\TOOLstarV, code={*_V}
  placeholder={A \TOOLstarV \psi},
  description={Applying operator/matrix to vector}}

\declare{macro=\TOOLstarS, code={*_S}
  placeholder={A \TOOLstarS S},
  description={Applying operator/matrix to subspace $S$}}

\declare{macro=\TOOLassign, code=\texttt{<-},
  placeholder={\texttt{$x$ <- $e$;}},
  description={Assignment (tool program syntax)}}

\declare{macro=\TOOLsample, code=\texttt{<\$},
  placeholder={\texttt{$x$ <\$ $e$;}},
  description={Sampling (tool program syntax)}}


\declare{macro=\TOOLqinit, code=\texttt{<q},
  placeholder={\texttt{$\qq_1,\dots,\qq_n$ <q $e$;}},
  description={Quantum initialization (tool program syntax)}}

\declare{macro=\TOOLpr, code=\xxx,
  placeholder={\texttt{Pr}[v:P(\rho)]},
  description={Isabelle/HOL constant for probability of $v=1$ after running $P$}}

%\declare{macro=\TOOLcdot, 
%  placeholder={A\cdot B},
%  description={Overloaded Isabelle/HOL constant for \texttt{timesOp}, \texttt{cblinfun\_apply}, \texttt{applyOpSpace}}}

% \declare{macro=\TOOLH, code=\texttt{H},
%   description={Hadamard (Isabelle/HOL constant)}}
% \declare{macro=\TOOLX, code=\texttt{X},
%   description={Pauli X (Isabelle/HOL constant)}}
% \declare{macro=\TOOLY, code=\texttt{Y},
%   description={Pauli Y (Isabelle/HOL constant)}}
% \declare{macro=\TOOLZ, code=\texttt{Z},
%   description={Pauli Z (Isabelle/HOL constant)}}

\declare{macro=\TOOLqvars,
  placeholder={\llbracket \qq_1,\dots,\qq_n\rrbracket},
  description={Typed tuple of quantum variables (Isabelle/HOL syntax)}}


\declare{macro=\TOOLfrqq, 
  placeholder={A\text\guillemotright Q},
  description={Applying operator $A$ to variables $Q$ (i.e., lifting $A$ to the whole memory via $Q$)}}

\declare{macro=\TOOLinq,  code={\in_\mathfrak q},
  placeholder={Q \TOOLinq S},
  description={Lifting subspace $S$ to the whole memory via $Q$}}

\declare{macro=\TOOLeqq,  code={=_\mathfrak q},
  placeholder={Q \TOOLeqq \psi},
  description={Lifting state $\psi$ to the whole memory via $Q$}}

\declare{macro=\TOOLqeq,
  placeholder={Q_1 \mathrel{\mathord\equiv\mathfrak q} Q_2,\
    Q_1\ \texttt{==q}\ Q_2},
  description={Isabelle/HOL syntax for quantum equality $Q_1\quanteq Q_2$}}
\declare{macro=\TOOLQeq,
  placeholder={\mathtt{Qeq[\qq_1,\dots,\qq_n=\qq'_1,\dots,\qq'_m]}},
  description={Isabelle/HOL syntax for quantum equality $\qq_1,\dots,\qq_n\quanteq \qq'_1,\dots,\qq'_m$}}

\declare{macro=\TOOLqvars,
  placeholder={\llbracket\qq_1,\dots,\qq_n\rrbracket},
  description={Typed tuple of quantum variables (Isabelle/HOL syntax)}}


% \declare{macro=\TOOLQEQ,
%   placeholder={\texttt{Qeq[$\qq_1,\dots,\qq_n$=$\qq'_1,\dots,\qq'_m$]}},
%   description={Isabelle/HOL syntax for quantum equality $\qq_1\dots\qq_n\quanteq\qq'_1\dots\qq'_m$}}

\declare{macro=\TOOLspacediv,
  placeholder={P\div \psi \text\guillemotright Q},
  description={Isabelle/HOL syntax for \texttt{space\_div} (related to $P\div \psi$)}}

\declare{macro=\floor, argspec=[1], code=\parenthesis\lfloor\rfloor{#1},
  placeholder=\floor x,
  description={$x$ rounded down to the next integer}}

\declare{macro=\sampleset, code=\in_{\scriptstyle{}\$},
  placeholder=x\sampleset M,
  description={$x$ uniformly sampled from $M$}}
\declare{macro=\AdvROR, code=\mathrm{Adv}_\mathrm{ROR},
  placeholder=\AdvROR^{A_1A_2}(\eta),
  description={Advantage of adversary $A_1,A_2$ in ROR-OT-CPA game with security parameter $\eta$}}
\declare{macro=\AdvIND, code=\mathrm{Adv}_\mathrm{IND},
  placeholder=\AdvIND^{A_1A_2}(\eta),
  description={Advantage of adversary $A_1,A_2$ in IND-OT-CPA game with security parameter $\eta$}}
\declare{macro=\AdvPRG, code=\mathrm{Adv}_\mathrm{PRG},
  placeholder=\AdvPRG^{A}(\eta),
  description={Advantage of adversary $A$ in PRG-OT-CPA game with security parameter $\eta$}}
\declare{macro=\Time, code=\mathrm{Time},
  placeholder=\Time(A),
  description={Worst-case runtime of $A$}}


\declare{macro=\SQCAP, code=\sqcap,
  placeholder={A\SQCAP B},
  description={Intersection of subspaces (Isabelle/HOL syntax)}}

\declare{macro=\SQCUP, code=\sqcup,
  placeholder={A\SQCUP B},
  description={Sum of subspaces (Isabelle/HOL syntax)}}

\declare{macro=\TOOLINF, code=\mathtt{INF},
  placeholder={\texttt{INF x:Z. e}},
  description={Intersection of family of subspaces (Isabelle/HOL syntax)}}

\declare{macro=\TOOLbot, code=\texttt{bot},
  description={Zero subspace (Isabelle/HOL syntax)}}

\declare{macro=\TOOLtop, code=\texttt{top},
  description={Full subspace (Isabelle/HOL syntax)}}


\declare{macro=\circexp, code=\circ_e,
  placeholder=a \circexp b,
  description={Composition of expressions $a,b$ as relations}}
\declare{macro=\circrel, code=\circ,
  placeholder=R_1 \circrel R_2,
  description={Composition of relations}}

\declare{macro=\setN, code=\mathbb N,
  description={Natural numbers}}

\declare{macro=\symdiff, code=\mathbin\Delta,
  placeholder=A\symdiff B,
  description={Symmetric difference of sets}}

%%% Local Variables: 
%%% mode: latex
%%% TeX-master: "manual"
%%% TeX-PDF-mode: t 
%%% End: 

%  LocalWords:  Powerset
